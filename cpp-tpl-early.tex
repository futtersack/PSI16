%% This is file `cpp-tpl.tex',
%% generated with the docstrip utility.
%%
%% The original source files were:
%%
%% template.dtx  (with options: `cpp')
%% 
%% $Id: template.dtx,v 1.80 2005/10/04 16:26:14 uwe Exp $
%% 2007-04-11 -MWL- aktualisiert
%% ====================================================================
\documentclass[cpp,a4paper,fleqn%
% ,finallayout% this activates the final column title entries
,earlyview% this adds the Early View tag on top of the final layout title page
]{w-art}
\usepackage{times,cite,w-thm}
% \usepackage{w-sidecapt}
%% By default the equations are consecutively numbered. This may be changed by
%% the following command.
%% \numberwithin{equation}{section}
%%
%% The definition of new theorem like environments.
%% Criterion
\theoremstyle{plain}
\newtheorem{criterion}{Criterion}
%% Condition
\theoremstyle{definition}
\newtheorem{condition}[theorem]{Condition}
%%
%% The usage of multiple languages is possible.
%% \usepackage{ngerman}% or
%% \usepackage[english,ngerman]{babel}
%% \usepackage[english,french]{babel}
\usepackage[]{graphicx}
%% 
\begin{document}
%%    The information for the title page will be placed between
%%    \begin{document} and \maketitle. The order of most entries
%%    is determined by the class file and can not be changed by
%%    rearranging them. The maketitle command follows after the
%%    abstract.
%%
%%    Most of the following commands will be completed by the publisher.
%%
%%    The copyrightyear is defined in the .clo file as the first argument
%%    of the copyrightinfo command. If the copyrightyear differs from that
%%    value it might be adjusted by the following definition:
%%
%% \renewcommand{\copyrightyear}{2007}% uncomment to change the copyrightyear.
%%
\DOIsuffix{theDOIsuffix}
%%
%% issueinfo for the header line
\Volume{46}
\Month{01}
\Year{2007}
%%
%%    First and last pagenumber of the article. If the option
%%    'autolastpage' is set (default) the second argument may be left empty.
\pagespan{1}{}
%%
%%    Dates will be filled in by the publisher. The 'reviseddate' and
%%    'dateposted' (Published online) entry may be left empty.
\Receiveddate{XXXX}
\Reviseddate{XXXX}
\Accepteddate{XXXX}
\Dateposted{XXXX}
%%
\keywords{List, of, comma, separated, keywords.}

%% \pretitle{Editor's Choice}

%% We have a short and a long form for the title. The short form
%% (optional argument) goes into the running head.

\title[Short title]{Long title}

%% Please do not enter footnotes or \inst{}-notes into the optional
%% argument of the author command. The optional argument will go into
%% the header.  If there is only one address the marker \inst{x} may be
%% omitted.

%% Information for the first author.
\author[F. Author]{First Author\inst{1,}%
  \footnote{Corresponding author\quad E-mail:~\textsf{x.y@xxx.yyy.zz},
            Phone: +00\,999\,999\,999,
            Fax: +00\,999\,999\,999}}
\address[\inst{1}]{First address}
%%
%%    Information for the second author
\author[S. Author]{Second Author\inst{1,2,}\footnote{Second author footnote.}}
\address[\inst{2}]{Second address}
%%
%%    Information for the third author
\author[T. Author]{Third Author\inst{2,}\footnote{Third author footnote.}}
%%
%%    \dedicatory{This is a dedicatory.}
\begin{abstract}
  This is an example input file.  Comparing it with the output it generates
  can show you how to produce a simple document of your own.
\end{abstract}
%% maketitle must follow the abstract.
\maketitle                   % Produces the title.

%% If there is not enough space inside the running head
%% for all authors including the title you may provide
%% the leftmark in one of the following three forms:

%% \renewcommand{\leftmark}
%% {First Author: A Short Title}

%% \renewcommand{\leftmark}
%% {First Author and Second Author: A Short Title}

%% \renewcommand{\leftmark}
%% {First Author et al.: A Short Title}

%% \tableofcontents  % Produces the table of contents.
\section{Introduction}

The class file \texttt{w-art.cls} represents an adaptation of the
\LaTeXe-standard class file \texttt{article.cls} and the \AmS{} class file
\texttt{amsart.cls} with the size option \texttt{10pt} to the specific
requirements of journal production at WILEY-VCH Verlag GmbH \& Co. KGaA,
Weinheim. It can be used through the \LaTeX-command
\begin{center}
    \verb+\documentclass[<abbr>, fleqn, other options]{w-art}+
\end{center}
where $<$abbr$>$ is an abbreviation of the journal name.

\begin{table}[h]
  \caption{Abbreviations for journal names.}
  \begin{tabular}{@{}lll@{}} \hline
    adp   & Ann. Phys. (Berlin)                  & Annalen der Physik \\
    cpp   & Contrib. Plasma Phys.                & Contributions to Plasma Physics \\
    gamm  & GAMM-Mitt.                           & GAMM-Mitteilungen \\
    mn    & Math. Nachr.                         & Mathematische Nachrichten \\
    mlq   & Math. Log. Quart.                    & Mathematical Logic Quarterly \\
    pamm  & PAMM $\cdot$ Proc. Appl. Math. Mech. & Proceedings in Applied Mathematics and Mechanics \\
    fdp   & Fortschr. Phys.                      & Fortschritte der Physik \\
    zamm  & ZAMM $\cdot$ Z. Angew. Math. Mech.   & Zeitschrift f\"{u}r Angewandte
                                                   Mathematik und Mechanik \\ 
\hline
\end{tabular}
  \label{tab:abbr}
\end{table}

  One difference to the standard layout is the indentation by \mbox{3\,cc} or
  \mbox{4\,cc} of floats (figures and tables) and mathematical environments
  (\verb+\[+\ldots\verb+\]+, \texttt{equation}, \ldots). To achieve this effect
  the new floats \texttt{vchtable} and \texttt{vchfigure} were added which are to be
  used in combination with \verb+\vchcaption+. The standard \texttt{table},
  \texttt{figure}, and \verb+\caption+ commands are nevertheless still working. So
  if there is the need to place a table or figure over the full width of the
  page these floats may still be used. In order to get short captions flushed
  left in contrast to the standard centered form the class loads internally the
  \texttt{caption2.sty} package by Harald Axel Sommerfeldt with the options
  \texttt{nooneline, small, bf}.

%%%%%%%%%%%%%%%%%%%%%%%%%%%%%%

\section{Required packages}
\label{required}
This class requires the standard \LaTeX\ packages \texttt{calc}, \texttt{color},
\texttt{sidecap}, and \texttt{caption2} and the \AmS-\LaTeX{}
packages.\footnote{If these packages are not part of your installation you may
  download them from the nearest CTAN server.}

%%%%%%%%%%%%%%%%%%%%%%%%%%%%%%

\subsection{New documentclass options}

\begin{description}

\item[referee] Prints the document with a larger amount of interline
  whitespace.
\end{description}

%%%%%%%%%%%%%%%%%%%%%%%%%%%%%%

\subsection{Floating objects -- figures and tables}

We have two different table environments: {\tt table} and {\tt vchtable}.  The
same holds true for figure: {\tt figure} and {\tt vchfigure}. The vch-types
including their captions (\verb+\vchcaption+) are typically leftindented by an amount
equal to the indentation of mathematical formulas.

For the caption layout the \texttt{caption2.sty} package is preloaded.

%%%%%%%%%%%%%%%%%%%%%%%%%%%%%%

\subsubsection{Tables}

The \LaTeX{} code for Table \ref{tab:2} is
\begin{verbatim}
\begin{table}
\caption{The caption inside a table environment.}
\label{tab:2}
\begin{tabular}{@{}lll@{}}
\hline
Description 1 & Description 2 & Description \\
\hline
Row 1, Col 1 & Row 1, Col 2 & Row 1, Col 3 \\
Row 2, Col 1 & Row 2, Col 2 & Row 2, Col 3 \\
\hline
\end{tabular}
\end{table}
\end{verbatim}
Please note the \verb+@{}+ entries at the begin and end of each \texttt{tabular}
column definition. These are used to remove the standard white space before the
first column and after the last column of the table. Thus, the material inside
the \texttt{tabular} environment will properly align at each side.

\begin{table}
\caption{The caption inside a table environment.}
\label{tab:2}
\begin{tabular}{@{}lll@{}}
\hline
Description 1 & Description 2 & Description \\
\hline
Row 1, Col 1 & Row 1, Col 2 & Row 1, Col 3 \\
Row 2, Col 1 & Row 2, Col 2 & Row 2, Col 3 \\
\hline
\end{tabular}
\end{table}

% \clearpage%%%%%%%%%%%%%%%%

The \LaTeX{} code for Table \ref{tab:3} (a vchtable) is
\begin{verbatim}
\begin{vchtable}
\vchcaption{The caption inside a vchtable environment.}
\label{tab:3}
\begin{tabular}{@{}lll@{}}
\hline
Description 1 & Description 2 & Description \\
\hline
Row 1, Col 1 & Row 1, Col 2 & Row 1, Col 3 \\
Row 2, Col 1 & Row 2, Col 2 & Row 2, Col 3 \\
\hline
\end{tabular}
\end{vchtable}
\end{verbatim}

\begin{vchtable}
\vchcaption{The caption inside a vchtable environment.}
\label{tab:3}
\begin{tabular}{@{}lll@{}}
\hline
Description 1 & Description 2 & Description \\
\hline
Row 1, Col 1 & Row 1, Col 2 & Row 1, Col 3 \\
Row 2, Col 1 & Row 2, Col 2 & Row 2, Col 3 \\
\hline
\end{tabular}
\end{vchtable}

%%%%%%%%%%%%%%%%%%%%%%%%%%%%%%

\subsubsection{Figures}

The \LaTeX{} code for Fig. \ref{fig:1} is
\begin{verbatim}
\begin{figure}[b]
\includegraphics[width=\linewidth, height=2cm]{empty}
\caption{The usual figure environment. It ...}
\label{fig:1}
\end{figure}
\end{verbatim}

\begin{figure}[b]
\includegraphics[width=\linewidth, height=2cm]{empty}
\caption{The usual figure environment. It may be used for figures spanning the
  whole page width.}
\label{fig:1}
\end{figure}

\begin{vchfigure}[b]
  \includegraphics[width=.5\textwidth,height=25mm]{empty}
\vchcaption{A vchfigure  environment with a vchcaption.
Figure and caption are leftindented.}
\label{fig:2}
\end{vchfigure}

The \LaTeX{} code for Fig. \ref{fig:2} (a vchfigure) is
\begin{verbatim}
\begin{vchfigure}[b]
  \includegraphics[width=.5\textwidth,height=25mm]{empty}
\vchcaption{A vchfigure  environment with a vchcaption.
Figure and caption are leftindented.}
\label{fig:2}
\end{vchfigure}
\end{verbatim}

The \LaTeX{} code for Figs. \ref{fig:3} and \ref{fig:4} is
\begin{verbatim}
\begin{figure}
\begin{minipage}{72mm}
\includegraphics[width=\linewidth,height=25mm]{empty}
\caption{Two figures side by side with different numbers.}
\label{fig:3}
\end{minipage}
\hfil
\begin{minipage}{65mm}
\includegraphics[width=\linewidth,height=25mm]{empty}
\caption{This is the second picture.}
\label{fig:4}
\end{minipage}
\end{figure}
\end{verbatim}

\begin{figure}
\begin{minipage}{72mm}
\includegraphics[width=\linewidth,height=25mm]{empty}
\caption{Two figures side by side with different numbers.}
\label{fig:3}
\end{minipage}
\hfil
\begin{minipage}{65mm}
\includegraphics[width=\linewidth,height=25mm]{empty}
\caption{This is the second picture.}
\label{fig:4}
\end{minipage}
\end{figure}

The \LaTeX{} code for Figs. \ref{fig:5}a and b is
\begin{verbatim}
\begin{figure}
\includegraphics[width=68mm,height=25mm]{empty}~a)
\hfil
\includegraphics[width=68mm,height=25mm]{empty}~b)
\caption{Two figures with one number. The figures
are referred to as \textbf{a} and \textbf{b}.}
\label{fig:5}
\end{figure}
\end{verbatim}

\begin{figure}
\includegraphics[width=68mm,height=25mm]{empty}~a)
\hfil
\includegraphics[width=68mm,height=25mm]{empty}~b)
\caption{Two figures with one number. The figures are referred to as \textbf{a}
and \textbf{b}.}
\label{fig:5}
\end{figure}

%%%%%%%%%%%%%%%%%%%%%%%%%%%%%%

In order to print a figure and a table side by side the
\verb+\setfloattype+ command is introduced. The \LaTeX{} code for
Fig. \ref{fig:8} and Table \ref{tab:4} is
\begin{verbatim}
\begin{figure}
\begin{minipage}{75mm}
\includegraphics[width=\linewidth,height=35mm]{empty}
\caption{Figure and table side by side. This is the picture.}
\label{fig:8}
\end{minipage}
\hfil
\begin{minipage}{65mm}
\setfloattype{table}
\caption{This is the table. ... .}
\label{tab:4}
\begin{tabular}{@{}lll@{}}
...
\end{tabular}
\end{minipage}
\end{figure}
\end{verbatim}

\begin{figure}
\begin{minipage}{75mm}
\includegraphics[width=\linewidth,height=35mm]{empty}
\caption{Figure and table side by side. This is the picture.}
\label{fig:8}
\end{minipage}
\hfil
\begin{minipage}{65mm}
\setfloattype{table}
\caption{This is the table. Picture and table are both numbered
independently.}
\label{tab:4}
\begin{tabular}{@{}lll@{}}
\hline
Description 1 & Description 2 & Description \\
\hline
Row 1, Col 1 & Row 1, Col 2 & Row 1, Col 3 \\
Row 2, Col 1 & Row 2, Col 2 & Row 2, Col 3 \\
\hline
\end{tabular}
\end{minipage}
\end{figure}


\subsubsection{Small figures and tables}

Single figures and tables less than 86mm wide should be typeset with their
captions on the right side, bottoms aligned. The command \verb+\sidecaption+
will achieve that for figures:
\begin{verbatim}
\begin{figure}[<float>]
\sidecaption
\includegraphics[<options>]{filename}%
\caption{Caption of a small figure.}
\label{fig:6}       % Give a unique label
\end{figure}
\end{verbatim}
where $<$float$>$ (optional) is the known floating position parameter. See
Fig. \ref{fig:sidecaption} for an example; the \LaTeX\ code for that is
\begin{verbatim}
\begin{figure}
\sidecaption
\includegraphics[width=50mm]{empty}%
\caption{Caption of a small figure.}
\label{fig:sidecaption}
\end{figure}
\end{verbatim}

\begin{figure}
\sidecaption
\includegraphics[width=50mm]{empty}%
\caption{Caption of a small figure.}
\label{fig:sidecaption}
\end{figure}


The \verb+\sidecaption+ macro expects just one \LaTeX{} box; so with more than
one simple picture file or some other structure, that has to be encapsulated in
a box structure:
\begin{verbatim}
\begin{figure}[<float>]
\sidecaption
\begin{minipage}{<boxwidth>}
...
\end{minipage}%
\caption{Caption of ... .}
\label{fig:X}       % Give a unique label
\end{figure}
\end{verbatim}
where $<$boxwidth$>$ is the width of the box that the \verb+\sidecaption+ macro
gets as argument.

\begin{figure}
\sidecaption
\begin{minipage}[b]{60mm}
\begin{minipage}[b]{33mm}%
\includegraphics[width=\linewidth,height=10mm]{empty}\\[3mm]%
\includegraphics[width=\linewidth,height=15mm]{empty}
\end{minipage}%
\hfill%
\begin{minipage}[b]{24mm}%
\includegraphics[width=\linewidth,height=25mm]{empty}
\end{minipage}%
\end{minipage}%
\caption{Caption of a \emph{complicated} small figure.}
\label{fig:couple}
\end{figure}

Note that virtually \emph{anything} could be typeset inside that box.
For example, in Fig.\,\ref{fig:couple} a couple of small items are assembled to
one picture; here is the \LaTeX\ code for that:
\begin{verbatim}
\begin{figure}
\sidecaption
\begin{minipage}[b]{60mm}
\begin{minipage}[b]{33mm}%
\includegraphics[width=\linewidth,height=10mm]{empty}\\[3mm]%
\includegraphics[width=\linewidth,height=15mm]{empty}
\end{minipage}%
\hfill%
\begin{minipage}[b]{24mm}%
\includegraphics[width=\linewidth,height=25mm]{empty}
\end{minipage}%
\end{minipage}%
\caption{Caption of a \emph{complicated} small figure.}
\label{fig:couple}
\end{figure}
\end{verbatim}
Note the (optional) \verb+[b]+ parameter which sets the base lines of the
minipages to their lower limits, so the outermost one positions exactly at the
correct height in respect to the side caption.

For tables \verb+\tabsidecaption+ has to be used instead:
\begin{verbatim}
\begin{table}[<float>]
\tabsidecaption
\begin{tabular}{@{}lll@{}}
\hline
Description 1 & Description 2 & Description \\
\hline
Row 1, Col 1 & Row 1, Col 2 & Row 1, Col 3 \\
Row 2, Col 1 & Row 2, Col 2 & Row 2, Col 3 \\
\hline
\end{tabular}
\caption{Caption of a small table.}
\label{tab:sidecaption}
\end{table}
\end{verbatim}
Note that for technical reasons the \verb+\caption+ command has to be placed
right \emph{after} the table when using this construct. See
Table~\ref{tab:sidecaption} for the result of the \LaTeX\ code fragment above.

\begin{table}[<float>]
\tabsidecaption
\begin{tabular}{@{}lll@{}}
\hline
Description 1 & Description 2 & Description \\
\hline
Row 1, Col 1 & Row 1, Col 2 & Row 1, Col 3 \\
Row 2, Col 1 & Row 2, Col 2 & Row 2, Col 3 \\
\hline
\end{tabular}
\caption{Caption of a small table.}
\label{tab:sidecaption}
\end{table}


%%%%%%%%%%%%%%%%%%%%%%%%%%%%%%

\section{Test of math environments}
Equations are always left-aligned. Therefore the option {fleqn} is used for the
documentclass command by default. Note that {fleqn} does not work with
unnumbered displayed equations written as \verb+$$ Ax =b $$+, so please use
\verb+\[ Ax=b \]+ or an \texttt{equation*} or \texttt{gather*} environment
instead.

By default the equations are consecutively numbered. This may be changed by
putting the following command inside the preamble
\begin{center}
  \verb+\numberwithin{equation}{section}+
\end{center}

The latex math display environment \verb+\[+\ldots\verb+\]+
\[
  \sum_{i=1}^{\infty} \frac{1}{i^2}
\]

An equation environment:
\begin{equation}
  \label{eq:eq}
  \sum_{i=1}^{\infty} \frac{1}{i^2}
\end{equation}
For more mathematical commands and environments please refer to the
document \texttt{*-tma.tex} and the documentation of the \AmS{} classes.

%%%%%%%%%%%%%%%%%%%%%%%%%%%%%%

\subsection{Some predefined theorem like environments}
Some predefined theorem like environments may be used by loading the package
\texttt{w-thm.sty}. This package will load by itself the package
\texttt{amsthm.sty}. So it will be easy to define new theorem- and
definition-like environments. For further details refer to the documentation of
the \texttt{amsthm.sty} package.
\begin{vchtable}[h]
  \vchcaption{Some predefined theorem like environments.}
\begin{tabular}{@{}lll@{}}
\hline
  environment            & caption     & theoremstyle  \\
\hline
  thm, theorem           & Theorem     & theorem  \\
  prop, proposition      & Proposition & theorem  \\
  lem, lemma             & Lemma       & theorem  \\
  cor, corollary         & Corollary   & theorem  \\
  axiom                  & Axiom       & theorem  \\
  defs, defn, definition & Definition  & definition  \\
  example                & Example     & definition  \\
  rem, remark            & Remark      & definition  \\
  notation               & Notation    & definition  \\
\hline
\end{tabular}
  \label{tab:thm}
\end{vchtable}
\begin{theorem}
  This is a theorem.
\end{theorem}
\begin{thm}
  Another theorem.
\end{thm}
\begin{proof}
  This is a proof.
\end{proof}
\begin{definition}
  This is a definition.
\end{definition}
\begin{proposition}
  This is a proposition.
\end{proposition}
\begin{lemma}
  This is a lemma.
\end{lemma}
\begin{corollary}
  This is a corollary.
\end{corollary}
\begin{example}
  This is an example.
\end{example}
\begin{remark}
  This is a remark.
\end{remark}

%%%%%%%%%%%%%%%%%%%%%%%%%%%%%%

\subsection{Definition of new theorem like environments}
Because \texttt{w-thm.sty} uses \texttt{amsthm.sty} the definition of new theorem like
environments will be done in the same manner as in the amsthm package.
The definition of
\begin{verbatim}
\theoremstyle{plain}
\newtheorem{criterion}{Criterion}
\theoremstyle{definition}
\newtheorem{condition}[theorem]{Condition}
\end{verbatim}
inside the preamble of the document will give the following envirenments.
\begin{criterion}
  This is a Criterion.
\end{criterion}
\begin{condition}
  This is a Condition.
\end{condition}

If the name of a predefined environment has to be changed it can be done
by e.g. typing
\begin{center}
\verb+\renewcommand{\definitionname}{Definitions}+
\end{center}
after the \verb+\begin{document}+ command.


%%%%%%%%%%%%%%%%%%%%%%%%%%%%%%

\subsection{Change marks}

Please use for changes requested by the referee the following colour change option:
\begin{changed}
  This is a text snippet marked as \emph{changed}. 
  This is
  done by enclosing it in an environment called \verb+changed+. Please note
  that in certain circumstances there might be small side effects such
  as make up deviations or additional blanks.
\end{changed}



\begin{acknowledgement}
  An acknowledgement may be placed at the end of the~article.
\end{acknowledgement}


\def\bstname{cpp}
\section*{Making bibliographies using Bib\TeX{} and \bstname.bst}


At the end of this file you see an example bibliography which
represents the journal�s reference
style\cite{bib1,bib2,bib3,bib4,bib5,bib6}. Please format your real
bibliography accordingly.

Optionally, you may generate your bibliography using Bib\TeX{}, with the
bib-style-file pss.bst from the template package in your \LaTeX{} search
path. To this end, replace the example database-file \texttt{\jobname.bib} with
your own existing one, or alternatively use it as a template for
generating your new database. The following code in your manuscript
source file enables Bib\TeX{}-functionality:
% 
\begin{quote}
\texttt{\textbackslash{}bibliographystyle\{\bstname\}}\\
\texttt{\textbackslash{}bibliography\{<}\emph{database-filename}\texttt{>\}}
\end{quote}
% 
You then need to run your manuscript source file through Bib\TeX{} using
the command 
% 
\begin{quote}
\mbox{\texttt{bibtex <}\emph{filename}\texttt{>}}
\end{quote}
% 
(most \TeX{} frontends have a shortcut for this) and afterwards
through \LaTeX{} twice, in order to get the correct label numbering.

\paragraph{Important:} Before
sending your manuscript source file to the publisher, remember to
transfer the actual, fully formatted bibliography contained in the
Bib\TeX{}-generated file \mbox{\texttt{<}\emph{filename}\texttt{>.bbl}}
to your manuscript source~file.


% Use this code if you wish to generate your bibliography with BibTeX;
% please replace first the string "demo" below with the name(s) of
% the BibTeX data base(s) you want to use.
% The resulting bibliography-output (the contents of the .bbl file)
% must be pasted into this file before submission.
% 
% \bibliographystyle{cpp}
% \bibliography{demo}
% 
% Replace the following example bibliography with your references
% before submission:
\begin{thebibliography}{[1]}

\bibitem{bib1}% 
 F.\,M. Firstauthorfamilyname, F.\,M. Secondauthorfamilyname, and
  C.~Lastauthorfamilyname,
 Abbreviatedjournalname \textbf{volume}, page (year).

\bibitem{bib2}% 
 F.~Examplename and  I.\,E. Anotherauthorname,
 phys. stat. sol. (a) \textbf{1}, 111 (2050).

\bibitem{bib3}% 
 A.~Firstauthorname,  B.~Secondauthorname,  and
  C.~Thirdauthorname,
Here Goes the Title of the Book (Publisher, City, year), p.\,111.

\bibitem{bib4}% 
 A.~Firsteditorname,  B.~Secondeditorname,  and
  C.~Thirdeditorname (eds.),
Here Goes the Title of the Edited Book (Wiley-VCH, Berlin, 2050), p.\,111.

\bibitem{bib5}% 
 D.~Contributorname,
 in: The Title of the Book, edited by The Name of the Editors, Followed by
  the Title of the Series of Books (Publisher, City, year), chap.~1.

\bibitem{bib6}% 
 A.~Lastbutnotleastname,
 Proceedings 1st Dummy Conference on Citation Formatting, City,
  Country, Part A (Publisher, City, year),  pp.\,1--11.

\end{thebibliography}
\end{document}